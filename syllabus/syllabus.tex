\documentclass{article}
\usepackage[margin=0.75in]{geometry}
\usepackage{multicol}
\usepackage{hyperref}
\usepackage{fancyvrb}
\usepackage{xcolor}

\newcommand{\nop}[1]{}

\begin{document}


\centerline{\Large \bf CSCI 321: Introduction to Computer Game Development}

\centerline{\bf Syllabus,  Spring, 2016}

\begin{description}

\item[Instructor:] Geoffrey Matthews, x3797,
 geoffrey dot matthews at wwu dot edu

\item[Office hours:] MTWF 10:00-10:50, CF 469

\item[Lectures:] MTWF 9:00-9:50am, CF 314


\item [Webpages:] \mbox{}\begin{itemize}
\item\url{https://wwu.instructure.com} 
\item\url{https://github.com/geofmatthews/csci321}
\end{itemize}


\item
[Content:]
This class is an introduction to the design, programming, ethics,
and business
of computer games.  Topics include:
\begin{multicols}{2}
  \begin{itemize}
\item Game design
  \item
 Game architecture
  \item
 Graphics
  \item
 Physics and simulations
  \item
 Game AI (artificial intelligence)
  \item
 Game Story
 \item Game Mechanics
  \item
 Social aspects of games
  \end{itemize}
\end{multicols}
\item
[Course objectives:]
At the end of this class the student should be able to:
%\begin{multicols}{2}
  \begin{itemize}
  \item Understand basic game architecture:
\\
-- Game loop\hfill
-- Game timing\hfill
-- Event oriented programming\hfill\mbox{}
  \item Understand basic nonplayer character AI:
\\
-- State oriented behavior\hfill
-- Autonomous motion\hfill
-- $A^*$ search\hfill
-- Goal driven behavior\hfill\mbox{}
  \item Understand the basics of 2D and 3D physics necessary for games:
\begin{itemize}
  \item Understand tradeoffs in numerical integration techniques
  \item Collision detection, resolution, and response
\end{itemize}
  \item Understand blitting and 2D graphics, transparency, animated sprites
  \item Understand the basics of 3D computer graphics
  \item Understand the basics of 3D animation, character modeling,
rigging and skinning
  \item Use Python and Pygame to produce a 2D game
  \item Use Blender gamekit to produce a 3D game, including 3D content
  \item Understand the elements of storytelling and drama:
character, conflict, and plot.
  \item Use Inform 7 to produce an interactive fiction game
  \item Understand the basics of the game industry today, job titles,
responsibilities, {\em etc.}
  \item Understand the social issues of games, such as
     game violence, addiction, griefing, and hacking.
  \end{itemize}
%\end{multicols}


\item [Exams:] One midterm and one final.  All exams will be
  closed book, except that you may bring two sheets of paper with
  information on them to be used during the exams.

\item [Quizzes:] Except for the first and last weeks of class and the
  week of the midterm, we will have weekly quizzes on Fridays.  Closed
  book and notes.  There will be no excused absences for quizzes, but
  the lowest two quiz scores will be dropped.

\item [Reading:] All students are expected to do the 
  reading assigned throughout the quarter in order to be prepared for
  the weekly quizzes and the exams.

\item [Games:] There will be three game programming
  assignments:  a 2D game in python and pygame, a 3D game in Blender3D,
  and an IF (interactive fiction) game in Inform 7.

\item [Game Journal:] All students are required to play at least 2
  hours of games every week.  You must keep a journal on games, hours,
  etc.  More detailed instructions will be handed out in class. Game
  journals are due every week except the last week (dead week) by
  midnight on Sunday

\item[Late work:] Submissions are due before midnight of the due
date. Anything turned in later than that time will be accepted with a
25\% penalty per day (or fraction of a day).  Anything more than 3
days late will not be accepted.  It is generally better to turn in
incomplete work on time than late work that is more finished.
It is your responsibility to
make sure all assignments are correctly submitted.

\item[Attendance policy:] Attendance is required.  Studies show that
  regular attendance is highly correlated with performance.  I will
  take attendance from time to time, and attendance will be worth
  5\% of your grade.

  You are responsible for all material covered in the lectures, books,
  handouts, or other assigned reading else.  If you miss a lecture,
  make sure you get notes from another student.

  If you have a well documented emergency (illness, military service,
  school sponsored athletic events, {\em etc.}) notify your instructor
  as soon as possible and present documentation (a note from your
  mother is not sufficient).  The instructor may, at his discretion,
  extend the due date for the assignment, schedule a make-up exam,
  or simply adjust your remaining scores to determine your grade.

\begin{multicols}{2}
\item [Grading:] Grades will be based on attendance, the three games
  you produce, your game journal, weekly quizzes, a midterm, and a final
  exam.  There are no extra credit opportunities for this class.
  Relative weighting of the various assessments and assignment of plus
  and minus is at the discretion of the instructor.

$A \ge 90 > B \ge 80 > C \ge 70 > D \ge 60 > F$

\columnbreak

\begin{centering}
\begin{tabular}{|rr|}\hline
Attendance & 5\%\\
 Game journal &    5\%\\
 Quizzes &         10\%\\
 Midterm &         10\%\\
 Final &           20\%\\
 2D pygame game &  20\%\\
 3D Blender game & 20\%\\
 Inform7 game &        10\%\\
\hline
\end{tabular}

\end{centering}
\end{multicols}


\item[Texts and Readings:] 


\href{http://www.ai-junkie.com/books/toc_pgaibe.html}
{\em Programming Game AI by Example} (in bookstore).
Other readings from online materials will be assigned as they come up.


\item [Schedule:]\mbox{}

\begin{tabular}{r|ccccccc|lll}                      
     &Su& Mo& Tu& We& Th& Fr& Sa&Topic & Deadlines \\\hline  
April
& 27 & 28 & 29 & 30 & 31 &  1 &  2 & Pygame\\
&  3 &  4 &  5 &  6 &  7 &  8 &  9 &  Pygame, Physics\\
& 10 & 11 & 12 & 13 & 14 & 15 & 16 &  Physics\\
& 17 & 18 & 19 & 20 & 21 & 22 & 23 &  Blender3D\\
& 24 & 25 & 26 & 27 & 28 & 29 & 30 &  Physics & 2D game due Wednesday\\
May
&  1 &  2 &  3 &  4 &  5 &  6 &  7 &  AI & Midterm Wednesday\\
&  8 &  9 & 10 & 11 & 12 & 13 & 14 &  Inform7, Story\\
& 15 & 16 & 17 & 18 & 19 & 20 & 21 &  AI & 3D game due Wednesday\\
& 22 & 23 & 24 & 25 & 26 & 27 & 28 &  AI\\
June
& 29 & {\color{gray}30} & 31 &  1 &  2 &  3 &  4 &  Social issues &IF game due Wednesday\\
&  5 &  6 &  7 &  8 &  9 & 10 & 11 &  Final Exam &Wednesday 8:00am\\
\end{tabular}

\item [Academic dishonesty:] Academic dishonesty policy and
  procedure is discussed in the University Catalog, Appendix D.  All
  students should read this section of the catalog.  Academic
  dishonesty consists of misrepresentation by deception or other
  fraudulent means.  In computer science courses this frequently takes
  the form of copying another's program, either a fellow student's
  program, or copying one from the web.  Due diligence should be
  exercised in the labs at all times, since both copying and letting
  someone else copy your program are equally culpable.  Do not walk
  away from your computer in the lab without logging out or locking
  the screen.  Do not print out code and then throw it away in the lab
  trash cans. Do not share files, even if it is just to ``show them
  something.''  Describe it in words, or talk to them in person, never
  share code.

\item [Collaboration:] Collaboration with your fellow students is
  a good way to learn.  Feel free to share ideas, solve problems, and
  discuss your programs with other students.  However, collaboration
  is {\em not} copying.  All code should be original.  Remember the
  \fbox{\bf Long Term Memory Rule}: After discussing homework with
  another student, each of you must destroy all written notes,
  pictures, files that you shared, erase the board, {\em
    etc.}.  After that, you must watch a rerun of {\em the Simpson's},
  play a round of ping-pong, go for a walk, or do something else
  unrelated, for half an hour.  Then you can take the knowledge you
  gained from another student and put it to work, since it is now not
  copying, but learning.  You have made it your own.

\end{description}

\end{document}
